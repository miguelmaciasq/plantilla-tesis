\chapter{Conceptos y Trabajos Previos}
\label{cap:conceptosprevios}
En este capítulo se introducen los conceptos teóricos necesarios y también se introducen las técnicas utilizadas 
en el desarrollo de este trabajo. En la sección~\ref{sec:conceptos} se presentan los conceptos teóricos 
básicos, luego en la sección~\ref{sec:trabajosprevios} se describe una recopilación de los trabajos previos
existentes hasta la actualidad relacionados con el presente trabajo.

\section{Conceptos teóricos}
\label{sec:conceptos}

\subsection{Lógica Difusa y funciones de pertenencia} 
\index{Lógica!Difusa}
En 1965 \cite{cit:zadeh} se aplicó la lógica multivaluada a la teoría de conjuntos, estableciendo la posibilidad de que los elementos pudieran tener diferentes grados de pertenencia a un conjunto.
La lógica difusa (fuzzy logic) refleja muy cercanamente la manera en que razonamos los humanos con reglas aproximadas. Esta es una extensión de la lógica clásica diseñada para permitir  razonamiento sobre conceptos imprecisos. La lógica difusa es una lógica multivaluada que permite una gradación continua en el valor de verdad de una proposición, al poder utilizar cualquier valor en el intervalo [0,1].




\subsection{Códigos de Clasificación de Conchas de Abanico} 
\label{sec:codigos}

\subsubsection{Definiciones básicas}
\begin{definicion}
Si {V} es un espacio vectorial. El espacio proyectivo {P(V)} de {V} es el conjunto de sub-espacios vectoriales 
unidimensionales de {V}.
\end{definicion}

\begin{definicion}
Si el espacio vectorial {V} tiene $n+1$ dimensiones, entonces {P(V)} es un espacio proyectivo de dimensión $n$. 
A un espacio proyectivo unidimensional se le conoce como \textbf{linea proyectiva} (linea de proyección), 
y a un espacio bidimensional se lo define como \textbf{plano proyectivo}.
\end{definicion}

\subsubsection{Subespacios lineales}
\begin{definicion}
Un subespacio lineal del espacio proyectivo {P(V)} es el conjunto de vectores subespaciales unidimensionales de 
un vector subespacial $U \subseteq V$
\end{definicion}


\subsection{Desplazamiento de Funciones} 

\subsection{Sistemas Expertos} 


\subsection{Redes Neuronales} 

\section{Trabajos previos}\index{Trabajos Previos}
\label{sec:trabajosprevios}
Hasta ahora en la literatura se han tratado diversos trabajos relacionados con
la estimación de distancias usando dispositivos tales como una cámara web y un
puntero láser. En el siglo pasado se comenzó a estudiar los fenómenos ocurrentes
en las imágenes digitales, en~\cite{Brown1971} el autor logró establecer la importancia de la
estimación de distancias, estableció un modelo de sistema para obtener medidas cercanas 
fotogramétricas para la obtención de medidas de estructuras\ldots
  